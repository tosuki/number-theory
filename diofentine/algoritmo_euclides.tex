\documentclass{article}

\usepackage{amssymb}
\usepackage{amsmath}

\title{Demonstração Algoritmo de Euclides}
\author{Carlos Henrique}
\date{Setembro 2025}

\begin{document}
    \maketitle

    \section{Introdução}
    O Algoritmo de Euclides é um método resolutivo de $mdc$ criado pelo famoso matematico Euclides. O propósito do algoritmo é encontrar o máximo divisor comum entre dois números inteiros. Para isso, ele usufrui das propriedades da divisão e multiplicação para dividir um problema em mini problemas e a partir deles, se acha o mdc.

    \section{Propriedades}
    \subsection{Coeficientes nulos}
    Uma das principais propriedades que é fundamental para que o algoritmo de Euclides funcione é a seguinte:
    \begin{equation}
        mdc(a, 0) = a
    \end{equation}
    Essa propriedade nos diz que quando tentamos encontrar o máximo divisor comum entre um número inteiro e um número nulo (igual a 0), o máximo divisor comum sempre será o número inteiro. Isso ocorre devido a natureza do 0, todo número é multiplo de 0. Portanto, o máximo sempre será o limite estabelecido pelo número não nulo.
    \begin{equation}
        \begin{split}
            mdc(10, 0) = 10\\
            mdc(25, 0) = 25\\
            mdc(0, 35) = 35
        \end{split}
    \end{equation}
    Essas propriedades é interessante quando trabalhamos com essa abordagem de resolução de problemas, pois ela facilita na hora de encontrar a solução para os problemas de grandezas de complexidade inferiores ao problema inicial. A demonstração dela é bem simples na verdade, se todo número é multiplo de 0, Portanto, essa afirmação se torna verdadeira:
    \begin{equation}
        \begin{split}
            0 * a = 0 \to \begin{vmatrix}
                10*0\\
                25*0\\
                0*35\\
                \end{vmatrix} = \begin{vmatrix}
                0\\
                0\\
                0
            \end{vmatrix}
        \end{split}
    \end{equation}
    Como você pode ver, todos os números são múltiplos de 0.

    \subsection{Divisibilidade entre coeficiente e resto}
    A segunda propriedade e a peça chave em segmentar o problema em problemas menos complexos, é essa. Quando trabalhos com divisão euclidiana, trabalhos dessa forma:
    \begin{equation}
        a = bq + r
    \end{equation}
    Onde:
    \begin{enumerate}
        \item $a =$ dividendo
        \item $b =$ divisor
        \item $q =$ quociente
        \item $r =$ resto
    \end{enumerate}
    A partir dessa equação, podemos manipular ela pra encontrar diferentes outras equações pra cada monómio.
    \begin{equation}
        \begin{split}
            a = bq+r\\
            r = a-bq\\
            q = \frac{a-r}{b} \\
        \end{split}
    \end{equation}
    (TODO $\to$ equação do divisor)

    Portanto, temos a seguinte propriedade:
    \begin{equation}
        mdc(a,b) = mdc(b,r) = mdc(r, r_2) = mdc(r_2, r_n) = mdc(r_n, 0) = r_n
    \end{equation}
    E é possivel provar essa propriedade usando a propriedade dos multiplos. Dessa forma:
    \begin{equation}
        \begin{split}
            \exists mdc(a,b) \to mdc(a,b) | a, mdc(a,b) | b\\
            mdc(a,b) | a, mdc(a,b) | b \to \begin{vmatrix}
                m*mdc(a,b)\\
                n*mdc(a,b)
            \end{vmatrix} = \begin{vmatrix}
                a\\
                b
            \end{vmatrix}
        \end{split}
    \end{equation}
    O primeiro passo seria provar que $mdc(a,b) | r$ também. Daria pra escrever isso de outra forma. 
    \begin{equation}
        mdc(a,b) | (a-bq)
    \end{equation}
    No entanto, $bq$ sempre irá gerar um multiplo de $b$. Portanto, pra facilitar a manipulação, presumimos que $q = 1$. Portanto, precisamos provar que na verdade $mdc(a,b) | (a-b)$. Chamamos de $r = a-b$.
    \begin{equation}
        \begin{split}
            a-b = r\\
            m*mdc(a,b) = a\ \ \ n*mdc(a,b) = b\\
            m*mdc(a,b) - n*mdc(a-b) = r\\
            mdc(a,b)*(a-b) = r\\
            (a-b) * mdc(a,b) = r*mdc(a,b) = r
        \end{split}
    \end{equation}
    Portanto, podemos dizer que a diferença entre os coeficientes também é um multiplo de $mdc(a,b)$. Portanto, podemos dizer que $mdc(a,b) | r$. No entanto, agora falta demonstrar que $b$ e $r$ também é multiplo de $a$. Porque se isso for verdade, podemos dizer que:
    \begin{equation}
        \begin{split}
            mdc(b,r) | b\\
            mdc(b,r) | r\\
            mdc(b,r) | a
        \end{split}
    \end{equation}
    Portanto, se conseguirmos provar isso e chegar nesse conflito de inequação, pois se $mdc(a,b) >= mdc(b,r)$ e $mdc(b,r) >= mdc(a,b)$. Portanto, $mdc(a,b) = mdc(b,r)$. E para provar que $mdc(b,r) | a$, precisamos provar que $a$ também é multiplo de $mdc(b,r)$. Portanto:
    \begin{equation}
        \begin{split}
            r = a -b\\
            r + b = r\\
            \\
            b = n*mdc(b,r)\\
            r = k*mdc(b,r)\\
            \\
            a = n*mdc(b,r) + k*mdc(b-r)\\
            a = mdc(b,r) * (n+k)
        \end{split}
    \end{equation}
    Assim, fica demonstrado que $a$ também é multiplo de $mdc(b,r)$. Dessa forma, podemos dizer que:
    \[mdc(b,r) = mdc(a,b)\]
    Porque $mdc(a,b)$ não pode ser maior que $mdc(b,r)$ enquanto $mdc(b,r)$ é maior que $mdc(a,b)$. Portanto, o que resta é eles serem iguais. A partir disso, isso aqui se torna verdade:
    \begin{equation}
        mdc(a,b) = mdc(b,r) = mdc(r, r_2) = mdc(r_2, r_n) = mdc(r_n, 0) = r_n
    \end{equation}
\end{document}
